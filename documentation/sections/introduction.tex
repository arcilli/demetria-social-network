\chapter{Introducere}

\textbf{Scop}

Prezentarea rezumativă a problematicii tratate:
\begin{itemize}
	\item Motivarea problemei (punerea în context, descrierea informală a unor abordări existente)
	\item descriera informală a soluției, prezentarea contribuțiilor autorului, structura lucrării
\end{itemize}
\textbf{Conținut}
\begin{itemize}
	\item motivarea temei/problemei abordate
	\item obiective urmărite/atinse
	\item relevanța rezultatelor/utilitatea aplicației
	\item structura lucrării
\end{itemize}
Responsabilitățile care trec în sarcina web-ului sunt din ce în ce mai mari pe măsură ce trece timpul. S-a trecut de la perioade în care aplicațiile putea fi hostate pe mașini clasice, iar acum serverele sunt clustere întregi. Avansul hardware a permis aplicațiilor software să poată permite stocarea de date redundante în detrimentul vitezei.

În acest context, s-a făcut trecerea de la arhitectura monolit la una orientată pe servicii. În prezent, se observă că deși prezintă un grad al decuplării ridicat, arhitectura bazată pe servicii (SOA - Service Oriented Architecture) are unele probleme, câteva dintre ele fiind:
\begin{itemize}
	\item aici 1
	\item aici 2
\end{itemize}

Conceptul de microserviciu a apărut în jurul anului 2014.

\textbf{Motivarea temei}


Conținut secțiune, exemplu citare \cite{hoare_csp}.
\textit{Cuvânt pentru} \textbf{glosar} \index{glosar}.	

\begin{definition}
	Conținut definiție
\end{definition}


\begin{lemma}
	Conținut lemă.
\end{lemma}

\begin{theorem}
	Conținut teoremă.
	\[
		e^{i * \pi} + 1 = 0
	\]
\end{theorem}

\begin{remark}
	Conținut observație
\end{remark}

Ecuație inline, $2^{10} = 1024 $

\begin{figure}[h]
	\begin{center}
			\includegraphics[width=4cm]{university.jpg}
	\end{center}
	\caption{Captură imagine}
\end{figure}


\subsection{Subsecțiune}

\subsubsection{Subsubsecțiune}