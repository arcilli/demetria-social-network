\chapter{Introducere}

\section{Scop}
Prezentarea rezumativă a problematicii tratate:
\begin{itemize}
	\item Motivarea problemei (punerea în context, descrierea informală a unor abordări existente)
	\item descriera informală a soluției, prezentarea contribuțiilor autorului, structura lucrării
\end{itemize}
\section{Conținut}
\begin{itemize}
	\item motivarea temei/problemei abordate
	\item obiective urmărite/atinse
	\item relevanța rezultatelor/utilitatea aplicației
	\item structura lucrării
\end{itemize}
\p


Conținut secțiune, exemplu citare \cite{hoare_csp}.
\textit{Cuvânt pentru} \textbf{glosar} \index{glosar}.	

\begin{definition}
	Conținut definiție
\end{definition}

\begin{lemma}
	Conținut lemă.
\end{lemma}

\begin{theorem}
	Conținut teoremă.
	\[
		e^{i * \pi} + 1 = 0
	\]
\end{theorem}

\begin{remark}
	Conținut observație
\end{remark}

Ecuație inline, $2^{10} = 1024 $

\begin{figure}[h]
	\begin{center}
			\includegraphics[width=4cm]{university.jpg}
	\end{center}
	\caption{Captură imagine}
\end{figure}


\subsection{Subsecțiune}

\subsubsection{Subsubsecțiune}